\documentclass[12pt]{article}
\usepackage[utf8]{inputenc}
\usepackage{float}
\usepackage{amsmath}

\usepackage[hmargin=3cm,vmargin=6.0cm]{geometry}
%\topmargin=0cm
\topmargin=-2cm
\addtolength{\textheight}{6.5cm}
\addtolength{\textwidth}{2.0cm}
%\setlength{\leftmargin}{-5cm}
\setlength{\oddsidemargin}{0.0cm}
\setlength{\evensidemargin}{0.0cm}

%misc libraries goes here
\usepackage{fitch}

\begin
	 {document}

\section*{Student Information } 
%Write your full name and id number between the colon and newline
%Put one empty space character after colon and before newline
Full Name :  Deniz Karakoyun  \\ 
Id Number :  2580678 \\ 

% Write your answers below the section tags
\section*{Answer 1}

\textbf{1.a)} To determine if the statement is a tautology or contradiction, we'll create a truth table for $(p \to q) \oplus (p \land \neg q)$:

\begin{table}[H]
\centering
\begin{tabular}{|c|c|c|c|c|c|}
\hline
p & q & $\neg q$ & $p \land \neg q$ & $p \to q$ & $(p \to q) \oplus (p \land \neg q)$ \\
\hline
T & T & F & F & T & T \\
T & F & T & T & F & T \\
F & T & F & F & T & T \\
F & F & T & F & T & T \\
\hline
\end{tabular}
\end{table}

The truth table shows that $(p \to q) \oplus (p \land \neg q)$ is always true (T) ; so it's a \textbf{tautology}.\\

\textbf{1.b)} To prove the propositional equivalence, we can use the given tables and laws: \\
 \begin{table}[H]
    \begin{tabular}{ccl|l}	
    $ p  \to  ((q \lor \neg p) \to r ) $ & $\equiv $ & $ p  \to (\neg(q \lor \neg p) \lor r)$ & By using (If elimination)\textbf{Table 7} \\
    & $\equiv $ & $ p \to ((\neg q \land p ) \lor r )$ & By using (De Morgan's laws) \textbf{ Table 6} \\
    & $\equiv $ & $ \neg p \lor ((\neg q \land p )\lor r)  $ & By using (If elimination) \textbf{Table 7} \\
    
    & $\equiv $ & $ (\neg p \lor (\neg q \land p))  \lor r$ & By using (Associatative Law) \textbf{Table 6} \\
    
    & $\equiv $ & $  ((\neg p \lor \neg q) \land (\neg p \lor p)) \lor r $ & By using (Distributive Law) \textbf{Table 6} \\
    
    & $\equiv $ & $ ((\neg p \lor \neg q) \land T) \lor r  $ & By using (Negation Law) \textbf{Table 6} \\
    & $\equiv $ & $ (\neg p \lor \neg q) \lor  r $ & By using (Identity Law)\textbf{Table 6} \\
    & $\equiv $ & $ \neg (\neg p \lor \neg q) \to  r$ & By using (If introduction)\textbf{Table 7}\\
    & $\equiv $ & $  ( p \land q) \to  r$ & By using (De Morgan's Law) \textbf{Table 6}\\
    \end{tabular}
\end{table}
Thus, $ p  \to  ((q \lor \neg p) \to r ) $ and  $ (p \land q) \to r  $ are logically equivalent.\\

\textbf{1.c)} Logical Equivalences: \\
\\\textbullet{} $(p \land q) \to r \equiv (\neg p \land \neg q) \lor r$: \textbf{False} \\
\textbullet{} $(p \land q) \lor (\neg p \land \neg q) \equiv p \oplus q$: \textbf{False}\\
\textbullet{} $(p \lor q) \land (p \land \neg q) \equiv p$: \textbf{False} \\
\textbullet{} $(p \lor q) \land (p \lor \neg q) \equiv p$: \textbf{True} \\
\textbullet{} $p \oplus q \equiv \neg (p \leftrightarrow q)$: \textbf{True} \\

\section*{Answer 2}

a) Player Can plays in a team that plays in league L.  \\
$\exists x (P(\text{Can}, x) \land T(x, L))$ \\
 \\
b) Every team in league S has at least one Turkish player. \\
$\forall x (T(x, \text{S}) \to \exists y (P(y, x) \land  N(y, \text{Turkish})))$ \\
 \\
c) Every team in league S has exactly one rival, which is also in league S. \\
$\forall x (T(x, S) \to \exists y (R(x, y) \land T(y, S) \land \forall z ((R(x, z) \land T(z, S) )\to (z = y))))$ \\
 \\
d) Team M has never won against a team with at least one English player. \\
$\forall x (W(\text{M}, x) \to \neg \exists y (N(y, \text{English}) \land P(y, x)))$ \\
 \\
e) Exactly two Turkish players play on team G. \\
$\exists x  \exists y((P(x, \text{G}) \land N(x, \text{Türk})) \land  (P(y, \text{G}) \land N(y, \text{Turkish})) \land x \neq y ) \land \forall z ((N(z, \text{Turkish}) \to \neg P(z, \text{G})) \land (z \neq x \land z \neq y))$ \\
 \\
f) There are some teams that play in more than one league. \\
$\exists x \exists y \exists z(T(x, y) \land T(x, z)  \land (z \neq y))$ \\

\section*{Answer 3}


\begin{table}[H]
	\centering
	\caption{Proof of $p \rightarrow q, (r \land s) \rightarrow p, (r \land \neg q) \vdash \neg s$}
	\begin{tabular}{lllclll}
	\hline
	\hline
	$1.$ & & & $p \rightarrow q$ & \textit{premise} & & \\
	$2.$ & & & $(r \land s) \rightarrow p$ & \textit{premise} & & \\
	$3.$ & & & $(r \land \neg q)$ & \textit{premise} & & \\
	
	

	\cline{2-7}
	$4.$ &\multicolumn{1}{|c}{} & & $s$ & \textit{assumption} & &\multicolumn{1}{c|}{} \\
	$5.$ &\multicolumn{1}{|c}{} & \multicolumn{1}{}{} & $r $ & $\land$e $3$ & &\multicolumn{1}{c|}{} \\
	$6.$ &\multicolumn{1}{|c}{} & \multicolumn{1}{}{} & $\neg q$ & $\land$e $3$ & &\multicolumn{1}{c|}{} \\
	$7.$ &\multicolumn{1}{|c}{} & \multicolumn{1}{}{} & $(r \land s)$ & $\land$i $4, 5$ & &\multicolumn{1}{c|}{} \\
	$8.$ &\multicolumn{1}{|c}{} & \multicolumn{1}{}{    } & $ (r \land s) \rightarrow p$ & \textit{copy 2} & &\multicolumn{1}{c|}{} \\
	$9.$ &\multicolumn{1}{|c}{} & \multicolumn{1}{}{} & $p$ & $\rightarrow$e $7,8$ & &\multicolumn{1}{c|}{} \\
	$10.$ &\multicolumn{1}{|c}{} & \multicolumn{1}{}{} & $p \rightarrow q$ & \textit{copy 1} & &\multicolumn{1}{c|}{} \\
	$11.$ &\multicolumn{1}{|c}{} & \multicolumn{1}{}{} & $q$ & $\rightarrow$e $9,10$ & &\multicolumn{1}{c|}{} \\
	
	$12.$ &\multicolumn{1}{|c}{} & & $\perp$ & $\neg$e $6,11$ & &\multicolumn{1}{c|}{} \\
	\cline{2-7}
	$13.$ & & & $\neg s$ & $\neg$i $4-12$ & & \\
	\end{tabular}
\end{table}






\section*{Answer 4}


\textbf{a)}  Converting sentences to logic:
  \begin{align*}
\textbullet{}&\text{Some students need to study for the exam in order to pass.} &&\equiv \exists x (P(x) \to S(x)) \\
\textbullet{} &\text{Every student passed the exam.} &&\equiv \forall x P(x)
\end{align*}\\


\textbf{b)} 
\begin{table}[H]
	\centering
	\caption{Proof of $ \forall x P(x) , \exists x (P(x) \to S(x)) \vdash \exists x S(x) $}
	\begin{tabular}{lllclll}
	\hline 
	\hline
		$1.$ & & & $ \forall x P(x) $ & \textit{premise} &  & \\ 
		
		$2.$ & & & $ \exists x (P(x) \to S(x))$ & \textit{premise} &  & \\ 
		
		
		\cline{3-6}
		 
		$3.$ &\multicolumn{1}{c}{}  & \multicolumn{1}{|c}{} & $P(c) \to S(c) $ & \textit{assumption} & \multicolumn{1}{l|}{} & \multicolumn{1}{c}{} \\ 
		
		
		$4.$ &\multicolumn{1}{c}{}  & \multicolumn{1}{|c}{} & $P(c)$ &$\forall $e $1$& \multicolumn{1}{c|}{} & \multicolumn{1}{c}{} \\ 
		
	    
	    $5.$ &\multicolumn{1}{c}{}  & \multicolumn{1}{|c}{} & $ S(c)$ &$\to$e $3,4$ & \multicolumn{1}{c|}{} & \multicolumn{1}{c}{} \\ 
	    
	   	    $6.$ &\multicolumn{1}{c}{}  & \multicolumn{1}{|c}{} & $\exists x S(x)$ &$\exists$i $5$ & \multicolumn{1}{l|}{} & \multicolumn{1}{}{} \\ 
	    \cline{3-6}
	    
	    $7.$ &\multicolumn{1}{c}{}  &  & $\exists x S(x)$ &$\exists$e $2,3-6$ &  & \multicolumn{1}{c}{} \\ 
	
	\end{tabular}
\end{table}


\end{document} 
