\documentclass[12pt]{article}
\usepackage[utf8]{inputenc}
\usepackage{float}
\usepackage{amsmath}
\usepackage{amsmath, amssymb}
\usepackage[hmargin=3cm,vmargin=6.0cm]{geometry}
%\topmargin=0cm
\topmargin=-2cm
\addtolength{\textheight}{6.5cm}
\addtolength{\textwidth}{2.0cm}
%\setlength{\leftmargin}{-5cm}
\setlength{\oddsidemargin}{0.0cm}
\setlength{\evensidemargin}{0.0cm}

%misc libraries goes here
\usepackage{fitch}

\begin{document}

\section*{Student Information } 
%Write your full name and id number between the colon and newline
%Put one empty space character after colon and before newline
Full Name :  Deniz Karakoyun\\
Id Number :  2580678\\

% Write your answers below the section tags
\section*{Answer 1}
\textbf{a)} 

\textbf{Proof by Mathematical Induction Base case (\( m = 4 \)):}
For \( m = 4 \), we have \( x_1, x_2, x_3, x_4 \) in \( C \), and \( \lambda_1, \lambda_2, \lambda_3, \lambda_4 \) such that \( \lambda_i \geq 0 \) and \( \sum_{i=1}^{4} \lambda_i = 1 \). We want to show that \( \sum_{i=1}^{4} \lambda_i x_i \) is in \( C \).

Using the convexity property, consider the points \( \lambda_1 x_1 + \lambda_2 x_2 \) and \( \lambda_3 x_3 + \lambda_4 x_4 \). Now, take \( t = \frac{1}{2} \), and apply the convexity property:

\[ \frac{1}{2}(\lambda_1 x_1 + \lambda_2 x_2) + \frac{1}{2}(\lambda_3 x_3 + \lambda_4 x_4) \]

By the convexity property, this point is in \( C \). But this is equal to \( \frac{1}{2} \sum_{i=1}^{4} \lambda_i x_i \). Therefore, \( \sum_{i=1}^{4} \lambda_i x_i \) is in \( C \).

\textbf{Inductive step:}
Assume the statement holds for \( m = k \) where \( k \geq 4 \). Now, we want to prove it for \( m = k+1 \).

We have \( x_1, x_2, \ldots, x_k, x_{k+1} \) in \( C \) and \( \lambda_1, \lambda_2, \ldots, \lambda_{k+1} \) with \( \lambda_i \geq 0 \) and \( \sum_{i=1}^{k+1} \lambda_i = 1 \). We want to show that \( \sum_{i=1}^{k+1} \lambda_i x_i \) is in \( C \).

Using the inductive hypothesis, we know that \( \sum_{i=1}^{k} \lambda_i x_i \) is in \( C \). Now, apply the convexity property to \( \sum_{i=1}^{k} \lambda_i x_i \) and \( x_{k+1} \) with \( t = \lambda_{k+1} \):

\[ \lambda_{k+1}(\sum_{i=1}^{k} \lambda_i x_i) + (1-\lambda_{k+1})x_{k+1} \]

By the convexity property, this point is in \( C \). But this is equal to \( \sum_{i=1}^{k+1} \lambda_i x_i \). Therefore, by mathematical induction, the statement holds for all \( m \geq 3 \).



\textbf{b)} 
\textit{Proof:}

Let $h = f \circ g$. To show that $h$ is convex, we need to prove that for any $x_1, x_2$ in the domain of \(g\) and any \(t\) in \([0, 1]\), the following inequality holds:
\[
h(tx_1 + (1-t)x_2) \leq th(x_1) + (1-t)h(x_2)
\]
Now, let's break it down:
\[
h(tx_1 + (1-t)x_2) = f(g(tx_1 + (1-t)x_2))
\]
Since \(g\) is convex, we know that:
\[
g(tx_1 + (1-t)x_2) \leq tg(x_1) + (1-t)g(x_2)
\]
Now, applying the convexity of \(f\) to the expression above:
\[
f(g(tx_1 + (1-t)x_2)) \leq f(tg(x_1) + (1-t)g(x_2))
\]
Now, using the convexity of \(f\), we have:
\[
f(tg(x_1) + (1-t)g(x_2)) \leq tf(g(x_1)) + (1-t)f(g(x_2))
\]
Now, substituting this back into the original inequality:
\[
h(tx_1 + (1-t)x_2) \leq tf(g(x_1)) + (1-t)f(g(x_2))
\]
Which is equivalent to:
\[
h(tx_1 + (1-t)x_2) \leq th(x_1) + (1-t)h(x_2)
\]
Thus, we've shown that \(h = f \circ g\) is convex.\\
\\


\textbf{c)}
\textit{Proof:}
\textbf{case 1)}
Assume $f$ is convex. We need to show that $S$ is convex and $g(t) = f(x + tv)$ is convex for all $t$ such that $x + tv \in S$.

1. \textbf{$S$ is Convex:}
   \begin{align*}
   &\text{For } x_1, x_2 \in S \text{ and } t \in [0, 1], \\
   &f(tx_1 + (1-t)x_2) \leq tf(x_1) + (1-t)f(x_2)
   \end{align*}

   Now, let $x = tx_1 + (1-t)x_2$. Since $f$ is convex, the inequality holds. This implies $x \in S$, showing $S$ is convex.

2. \textbf{Convexity of $g(t)$:}
   \begin{align*}
   &\text{Consider } g(t) = f(x + tv). \text{ We want to show it is convex for } t \text{ such that } x + tv \in S. \\
   &\text{Let } y_1 = x + tv_1 \text{ and } y_2 = x + tv_2 \text{ where } t_1, t_2 \in \mathbb{R}. \\
   &\text{By convexity of } f: \\
   &f(ty_1 + (1-t)y_2) \leq tf(y_1) + (1-t)f(y_2) \\
   &\text{Substitute } y_1 \text{ and } y_2: \\
   &f(t(x + tv_1) + (1-t)(x + tv_2)) \leq t f(x + tv_1) + (1-t) f(x + tv_2) \\
   &\text{Simplify:} \\
   &f(x + t(v_1 + v_2)) \leq t f(x + tv_1) + (1-t) f(x + tv_2) \\
   &\text{Set } w = v_1 + v_2: \\
   &f(x + tw) \leq t f(x + tv_1) + (1-t) f(x + tv_2)
   \end{align*}

This shows that $g(t) = f(x + tv)$ is convex for all $t$ such that $x + tv \in S$.\\
\textbf{case 2)}
Now, assume $S$ is convex, and $g(t) = f(x + tv)$ is convex for all $t$ such that $x + tv \in S$. We want to show that $f$ is convex.

Given $x_1, x_2 \in S$ and $t \in [0, 1]$, let $x = tx_1 + (1-t)x_2$. Since $S$ is convex, $x \in S$. Now, consider $g(t) = f(x + tv)$. By convexity of $g(t)$, we have:
\[f(x + tv) \leq tf(x_1) + (1-t)f(x_2)\]

Substitute $x = tx_1 + (1-t)x_2$:
\[f(tx_1 + (1-t)x_2 + t(vx_1 - x_1) + (1-t)(vx_2 - x_2)) \leq t f(x_1) + (1-t) f(x_2)\]

Simplify the expression:
\[f(tx_1 + (1-t)x_2) \leq t f(x_1) + (1-t) f(x_2)\]

This shows that $f$ is convex, completing the proof.

\section*{Answer 2}


\textbf{a)}
\begin{itemize}
    \item \( X \) is in \( \Sigma \): This statement is true because \( X - X = \varnothing \) (which is finite). Therefore, \( X \) is in the set.
    \item Closed under complementation: If \( A \) is in the set, then \( X - A \) is either finite or \( \varnothing \). The complement of \( A \) is \( X - (X - A) = A \), which is also in the set. This property is satisfied.
    \item Closed under countable unions: Let \( A_1, A_2, \ldots \) be sets in the set, meaning \( X - A_i \) is either finite or \( \varnothing \) for each \( i \). The union \( A = A_1 \cup A_2 \cup \ldots \) will also satisfy \( X - A \) being either finite or \( \varnothing \). Therefore, this set is closed under countable unions.
\end{itemize}

Conclusion: The set in part (a) satisfies all three properties, making it a \( \sigma \)-algebra on \( X \).

\textbf{b)}
\begin{itemize}
    \item \( X \) is in \( \Sigma \): This is true because \( X - X = \varnothing \) (which is countable). Therefore, \( X \) is in the set.
    \item Closed under complementation: If \( A \) is in the set, then \( X - A \) is either countable or \( X \). The complement of \( A \) is \( X - (X - A) = A \), which is also in the set. This property is satisfied.
    \item Closed under countable unions: Let \( A_1, A_2, \ldots \) be sets in the set, meaning \( X - A_i \) is either countable or \( X \) for each \( i \). The union \( A = A_1 \cup A_2 \cup \ldots \) will also satisfy \( X - A \) being either countable or \( X \). Therefore, this set is closed under countable unions.
\end{itemize}

Conclusion: The set in part (b) satisfies all three properties, making it a \( \sigma \)-algebra on \( X \).

\textbf{c)}
\begin{itemize}
    \item \( X \) is in \( \Sigma \): This is true because \( X - \varnothing = X \) (which is infinite). Therefore, \( X \) is in the set.
    \item Closed under complementation: If \( A \) is in the set, then \( X - A \) is either infinite, \( \varnothing \), or \( X \). The complement of \( A \) is \( X - (X - A) = A \), which is also in the set. This property is satisfied.
    \item Closed under countable unions: Let \( A_1, A_2, \ldots \) be sets in the set, meaning \( X - A_i \) is either infinite, \( \varnothing \), or \( X \) for each \( i \). The union \( A = A_1 \cup A_2 \cup \ldots \) will also satisfy \( X - A \) being either infinite, \( \varnothing \), or \( X \). Therefore, this set is closed under countable unions.
\end{itemize}

Conclusion: The set in part (c) satisfies all three properties, making it a \( \sigma \)-algebra on \( X \).




\section*{Answer 3}

\textbf{a)}
\textit{Proof:}
The congruence  $ ax \equiv  b\pmod{p} $ has a solution for x if and only if $ x \equiv c \pmod{q} $ for some $c \in \mathbb{Z}$ and some $q \in \mathbb{N}_0/{0}$\\
Let's build the equalities : 
\begin{align}
    1. & \quad ax \equiv b \pmod{p} \tag{1} \\
    2. & \quad x \equiv c \pmod{q} \tag{2}
\end{align}
Putting the first equation into the second one we get :
$$ a(qz + c) - pk \equiv b$$
Since $gcd(a,p) \neq 0$ we can divide both side with $gcd(a,p)$. In addition, as $ gcd(a,p) | a $ and $ gcd(a,p) | p $ the left hand side of the equation will be some number in  $ \mathbb{Z}$ so the right hand side of the equation must be in $\mathbb{Z}$ so $gcd(a,p)|b$ 






\textbf{b) Proof:}

We know that $\gcd(p_1, p_2) = 1$. By Euclid's theorem, there exist integers $u$ and $v$ such that $up_1 + vp_2 = 1$.

Now, let's consider the solution $x = u \cdot a_2 \cdot p_1 + v \cdot a_1 \cdot p_2$.

Checking this solution modulo $p_1$:
\[x \equiv u \cdot a_2 \cdot p_1 + v \cdot a_1 \cdot p_2 \equiv v \cdot a_1 \cdot p_2 \equiv a_1 \cdot (vp_2) \equiv a_1 \cdot 1 \equiv a_1 \pmod{p_1}.\]

Checking this solution modulo $p_2$:
\[x \equiv u \cdot a_2 \cdot p_1 + v \cdot a_1 \cdot p_2 \equiv u \cdot a_2 \cdot p_1 \equiv a_2 \cdot (up_1) \equiv a_2 \cdot 1 \equiv a_2 \pmod{p_2}.\]

So, we have shown that there exists c and q satisfying $x \equiv c \pmod{q}$ when $\gcd(p_1, p_2) = 1$. Therefore  $x$ satisfies both congruences, and we've shown that a solution exists when $\gcd(p_1, p_2) = 1$.
\\



\textbf{c)}
The system of congruences:


$$a_2x \equiv b_2 \pmod{p_2}$$ \\
$$a_1x \equiv b_1 \pmod{p_1} $$\\
$$a_1x \equiv b_1 \pmod{p_1} $$\\
$$a_2x \equiv b_2 \pmod{p_2} $$\\
\vdots \\
$$a_kx \equiv b_k \pmod{p_k}$$


has a solution for \(x\) of the form \(x \equiv c \pmod{\Pi}\), where \(\Pi = p_1p_2\ldots p_k\) and \(c \in \mathbb{Z}\), according to the Chinese Remainder Theorem.

\textbf{Explanation:}

The Chinese Remainder Theorem states that if \(p_1, p_2, \ldots, p_k\) are pairwise coprime, then the system of congruences has a unique solution modulo \(\Pi = p_1p_2\ldots p_k\).

In this case, the system is written as:


$$a_2x \equiv b_2 \pmod{p_2}$$ \\
$$a_1x \equiv b_1 \pmod{p_1}$$ \\
$$a_1x \equiv b_1 \pmod{p_1}$$ \\
$$a_2x \equiv b_2 \pmod{p_2}$$ \\
\vdots \\
$$a_kx \equiv b_k \pmod{p_k}$$


If \(x_1, x_2, \ldots, x_k\) are the solutions to each individual congruence, then the solution \(x\) modulo \(\Pi\) is given by:

\[ x \equiv c \pmod{\Pi} \]

where \(c = a_1x_1M_1 + a_2x_2M_2 + \ldots + a_kx_kM_k\), and \(M_i = \frac{\Pi}{p_i}\).

Therefore, the solution for \(x\) exists and is of the form \(x \equiv c \pmod{\Pi}\).


\section*{Answer 4}
\textbf{a)}
Given \( X = \{a, b, \ldots, z\} \) with \( |X| = 29 \), we want to determine if \( \prod_{i \in \mathbb{Z}^+} X_i \) is countable.

Each \( X_i \) is equal to \( X \), and there are countably many sets being multiplied together. We can represent each element in the Cartesian product as an infinite sequence of elements from \( X \). For example, an element in \( X_1 \times X_2 \times X_3 \times \ldots \) could be represented as \( (a, b, c, \ldots) \).

Now, we can use the fact that the Cartesian product of countably many countable sets is countable. Let \( A_i \) be the set of all sequences of length \( i \) with elements from \( X \). Since \( X \) is countable, each \( A_i \) is countable. The Cartesian product \( A_1 \times A_2 \times A_3 \times \ldots \) is then countable.

So, \( \prod_{i \in \mathbb{Z}^+} X_i \) is countable.


\textbf{b)}Given a family of countably infinite sets \( \{Y_i\}_{i \in \mathbb{Z}^+} \), we want to determine if \( \bigcup_{i \in \mathbb{Z}^+} Y_i \) is countable.

Let's denote the elements of \( Y_i \) as \( y_{i,1}, y_{i,2}, \ldots \). Now, we can list the elements of the union as follows:

\[ \bigcup_{i \in \mathbb{Z}^+} Y_i = \{ y_{1,1}, y_{1,2}, \ldots, y_{2,1}, y_{2,2}, \ldots, y_{3,1}, y_{3,2}, \ldots, \ldots \} \]

Since each \( Y_i \) is countable, we can create a bijection between \( \mathbb{N} \) and the elements of each \( Y_i \). Now, we can create a bijection from \( \mathbb{N} \) to the union of all \( Y_i \) by interleaving the elements.

Therefore, \( \bigcup_{i \in \mathbb{Z}^+} Y_i \) is countable.


\end{document}
