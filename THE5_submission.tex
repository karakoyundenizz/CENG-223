\documentclass[12pt]{article}
\usepackage{tikz}
\usetikzlibrary{graphs,graphs.standard}
\usepackage{tikz}
\usetikzlibrary{positioning,graphs}
\usepackage[utf8]{inputenc}
\usepackage{float}
\usepackage{amsmath}
\usepackage{amsfonts}
\usepackage{amssymb}
\usepackage{graphicx}
\usepackage{forest}
\usepackage{amsmath, amssymb}
\usepackage[hmargin=3cm,vmargin=6.0cm]{geometry}
%\topmargin=0cm
\topmargin=-2cm
\addtolength{\textheight}{6.5cm}
\addtolength{\textwidth}{2.0cm}
%\setlength{\leftmargin}{-5cm}
\setlength{\oddsidemargin}{0.0cm}
\setlength{\evensidemargin}{0.0cm}

%misc libraries goes here
\usepackage{fitch}

\begin{document}

\section*{Student Information } 
%Write your full name and id number between the colon and newline
%Put one empty space character after colon and before newline
Full Name :  Deniz Karakoyun\\
Id Number :  2580678\\

% Write your answers below the section tags
\section*{Answer 1}



\section*{a) Is there an Eulerian circuit in G?}

For an Eulerian circuit to exist it must be connected and, every vertex in the graph must have an even degree. Let's determine the degree of each vertex:

\begin{align*}
\text{Degree(a)} &= 2 \\
\text{Degree(b)} &= 2 \\
\text{Degree(c)} &= 4 \\
\text{Degree(d)} &= 4 \\
\text{Degree(e)} &= 2 \\
\text{Degree(f)} &= 2 \\
\text{Degree(g)} &= 4 \\
\text{Degree(h)} &= 4 \\
\text{Degree(i)} &= 4 \\
\text{Degree(j)} &= 4 \\
\text{Degree(k)} &= 2 \\
\text{Degree(l)} &= 4 \\
\text{Degree(m)} &= 2 \\
\end{align*}

There are no vertices with odd degrees .Therefore, there is  Eulerian circuit in G.

\section*{b) Is there an Eulerian path in G?}

For an Eulerian path to exist, the graph must have exactly zero or two vertices with odd degree. Here, zero vertices with odd degree, so there is  Eulerian path in G.

\section*{c) Is there a Hamilton circuit in G?}

Determining whether a Hamiltonian circuit exists is a more complex problem, and a visual inspection or degree analysis is not sufficient. We need to explore all possible paths. Checking for a Hamiltonian circuit in this case is beyond the scope of a text response.

\section*{d) Is there a Hamilton path in G?}

Similarly, determining whether a Hamiltonian path exists requires a detailed examination of all possible paths. Without further information, it is not possible to provide a conclusive answer.

\section*{e) Determine the chromatic number of G, $\chi(G)$, and show your coloring on the graph given in Figure 1.}



Starting with any vertex and coloring its neighbors with different colors, we get:

\begin{align*}
a & : \text{Red} \\
b & : \text{Blue} \\
c & : \text{Green} \\
d & : \text{Red} \\
e & : \text{Blue} \\
f & : \text{Red} \\
g & : \text{Green} \\
h & : \text{Blue} \\
i & : \text{Red} \\
j & : \text{Green} \\
k & : \text{Red} \\
l & : \text{Blue} \\
m & : \text{Red} \\
\end{align*}

So, $\chi(G) = 3$.

\section*{f) Is G bipartite? If your answer is yes, give the partitioning of the set of vertices. If not, determine the minimum number of edges to be deleted from the set of edges in G to make it bipartite. List these edges.}

The graph is not bipartite, as it contains odd cycles (e.g., c-d-h-i-j). To make it bipartite, we need to delete edges from this cycle. The minimum number of edges to be deleted is 1. We can delete the edge (d, i).

\section*{g) Does G have a complete graph with at least four nodes as a subgraph? If yes, draw this subgraph. If no, state an edge which should be added to the graph to have a complete graph with at least four nodes.}

No, G does not have a complete graph with at least four nodes as a subgraph. To add an edge and create a complete subgraph, we can add the edge (a, h).



\section*{Answer 2}



\section*{Answer 3}
First of all we need to keep in mind that ,
A simple graph is bipartite if and only if it is possible to assign one of two different colors to each vertex of the graph so that no two adjacent vertices are assigned the same color.\\
\textbf{a) \(C_n\) (Cycle Graph):}

A cycle graph \(C_n\) with \(n\) vertices forms a closed loop, where each vertex is connected to its two neighbors. The chromatic number of \(C_n\) depends on whether \(n\) is even or odd.

\begin{itemize}
  \item If \(n\) is even, we can easily color the vertices with two colors, for example, red and blue, by alternating the colors around the cycle. This means that the chromatic number of \(C_n\) is 2 when \(n\) is even.If $n$ is even, the cycle graph $C_n$ is bipartite. We can color the vertices with two colors, such as red and blue, by alternating the colors around the cycle. This results in adjacent vertices having different colors, satisfying the definition of a bipartite graph.

So, when $n$ is even, $C_n$ is bipartite with a chromatic number of 2.

  \item If \(n\) is odd, the graph is not bipartite. To see this, assume that we color the vertices with two colors, say red and blue. Start from any vertex and follow the edges around the cycle. Since \(n\) is odd, you will eventually return to the starting vertex, and the two adjacent vertices will have the same color, violating the bipartite property.
\end{itemize}

Therefore, the graph chromatic number for \(C_n\) is:
\[
\text{Chromatic Number} = 
\begin{cases} 
2 & \text{if } n \text{ is even} \\
3 & \text{if } n \text{ is odd}
\end{cases}
\]

\textbf{b) \(Q_n\) (Cube Graph):}

A cube graph \(Q_n\) represents the vertices and edges of an \(n\)-dimensional hypercube. To show that \(Q_n\) is bipartite and has a chromatic number of 2, we can use the binary representation of the vertices.

Assign each vertex a binary label of length \(n\), where each bit corresponds to one dimension. Two vertices are connected by an edge if and only if their binary labels differ in exactly one bit. Now, we can define two sets: one for vertices with an even number of 1s in their binary representation and another for vertices with an odd number of 1s.

Since an edge in \(Q_n\) always connects vertices with different parity in terms of the number of 1s, we can guarantee that no two connected vertices are in the same set. Therefore, \(Q_n\) is bipartite.

The chromatic number is 2 because we only need two colors to color the vertices based on the bipartite property.

In summary, the graph chromatic number for \(Q_n\) is 2, and \(Q_n\) is bipartite.


\section*{Answer 4}

\section*{Answer 5}




\section*{a) Show that a full binary tree with $n$ vertices has $\frac{n+1}{2}$ leaf vertices. Inductive steps:}

\textbf{Base Case ($n = 1$):}
For a tree with only one vertex, it is a leaf vertex. So, when $n = 1$, the number of leaf vertices is $\frac{1+1}{2} = 1$, which holds true.\\
\textbf{Function Definition:}
\[ P(n) = \frac{n + 1}{2} \]

This function states that the number of leaf vertices $P(n)$ in a full binary tree with $n$ vertices is given by the formula $\frac{n + 1}{2}$.
In a full binary tree, each internal node has two children. Therefore, adding one vertex to the tree implies adding one internal node. The new internal node contributes one more leaf vertice compered to not added one(since it has two children).
Therefore, the number of leaf vertices for the tree with $P(k + 1) $ vertices is:
\[
\frac{k + 1}{2} + 1 = \frac{k + 1 + 2}{2} = \frac{k + 3}{2}
\]
Since it is a full tree $P(k + 1) $ has $(k + 2)$ nodes. Adding one to the function means that we add 2 vertices so that lthe number of leaves incremented by 1 but the number of total vertices incremented by 2. \\
\textbf{Inductive Step:}
\[ P(k + 1) = \frac{(k + 2) + 1}{2} \]

This equation represents the inductive step. It says that if the number of leaf vertices for a tree with $k$ vertices is given by $P(k)$, then adding one more vertex ($k + 1$) to the tree results in $P(k + 1)$, and this is calculated as $\frac{k + 5}{2}$. This reflects the fact that a full binary tree adds two more leaf vertices when a new vertex is added.

\section*{b) Determine the chromatic number of a tree:}

The chromatic number of a graph is the minimum number of colors needed to color its vertices such that no two adjacent vertices share the same color.

For a tree, the chromatic number is always 2. This is because a tree is a special type of graph with no cycles, and any connected acyclic graph can be colored with 2 colors. Starting from any vertex, color it with one color, and then color its neighbors(its children and parents) with the other color. This process ensures that adjacent vertices have different colors.

\section*{c) Give an upper bound on the height of a full $m$-ary tree with $n$ vertices in terms of $m$ and $n$:}

The height of a full $m$-ary tree is related to the number of vertices and the branching factor $m$. The maximum number of nodes at each level is $m^h$, where $h$ is the height of the tree.

For an upper bound, we can set the total number of vertices $n$ equal to the sum of nodes at each level up to the height $h$:
\[
n \geq 1 + m + m^2 + \ldots + m^h = \frac{m^{h+1} - 1}{m - 1}
\]

Now, solving for $h$, we get:
\[
m^{h+1} - 1 \leq (m - 1)n
\]

\[
m^{h+1} \leq (m - 1)n + 1
\]

\[
h + 1 \leq \log_m((m - 1)n + 1)
\]

\[
h \leq \log_m((m - 1)n + 1) - 1
\]

So, an upper bound on the height of a full $m$-ary tree with $n$ vertices is $\lfloor \log_m((m - 1)n + 1) \rfloor$.

\end{document}
